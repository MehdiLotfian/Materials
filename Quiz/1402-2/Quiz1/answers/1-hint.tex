\subsection*{الف}

زبان منظم مورد نظر باید 2 ویژگی داشته باشد:
\begin{enumerate}
	\item فقط در انتهای رشته تعداد صفر یا بیشتر b مشاهده شود
	\item طول رشته فرد باشد

\end{enumerate}

برای این که شرط اول برقرار باشد، ساختار کلی عبارت منظم ما باید به صورت 
{\large 
$(a)^*(b)^*$
}باشد.

برای برقراری شرط دوم، باید به این نکته توجه کنیم که رشته به طول فرد از یک رشته به طول زوج ساخته می‌شود، پس در ابتدا یک رشته به طول زوج می‌سازیم، سپس آن را تبدیل به رشته‌ای به طول فرد می‌کنیم.

عبارت منظم رشته‌ی به طول زوج ما به این صورت است:

\setLTR

{\large 
	$(aa)^*(bb)^*$
}

\setRTL

حال برای این که طول رشته را فرد کنیم، کافیست یک کارکتر به آن اضافه کنیم. این کارکتر می‌تواند ‌a یا b باشد و حتماً باید قبل از {\large $(bb)^*$} باشد تا شرط اول نقض نشود.

پس زبان منظم ما به این صورت خواهد بود:

\setLTR

	{\large 
		$(aa)^*(a+b)(bb)^*$
	}
\setRTL



\subsection*{ب}

می‌دانیم که عبارت منظم {\large $(a+b)^*$} همه‌ی رشته‌های متشکل از الفبای $\{a,b\}$ را می‌پذیرد. حال عبارت منظم داده شده را ساده می‌کنیم:
\setLTR

{\large 

$((a)^*(a+b)^*(b)^*)^* = ((a)^0(a+b)^*(b)^0)^* = ((a+b)^*)^* = (a+b)^*$

}

\setRTL



بنابراین عبارت منظم ما تمامی رشته‌های متشکل از الفبای $\{a,b\}$ را می‌پذیرد.





